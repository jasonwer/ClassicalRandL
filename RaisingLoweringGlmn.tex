\documentclass[12pt]{article}
%
\usepackage{amssymb,amsmath,cite,mathdots}
%\usepackage{empheq}
%\usepackage{bm}%
%\usepackage{accents}


\setlength{\textheight}{23cm}
\setlength{\textwidth}{16cm}
\setlength{\topmargin}{0cm}
\setlength{\headheight}{0pt}
\setlength{\oddsidemargin}{0pt}
\setlength{\evensidemargin}{0pt}
%\setlength{\unitlength}{0.7mm}
%
\def\beq{\begin{equation}}
\def\eeq{\end{equation}}
\def\bea{\begin{eqnarray*}}
\def\eea{\end{eqnarray*}}
\def\nn{\nonumber}
\def\cga{\mathfrak{g}_{\ell}}
\def\bra#1{\left\langle #1\right|}
\def\ket#1{\left| #1\right\rangle}
\def\bracket#1#2{\left\langle #1 | #2 \right\rangle}
\def\NN{ {\mathbb Z}^{+} }
\def\N0{ {\mathbb Z}_{+} }
\def\v0{ |d,r) }
\def\del#1{\partial_{#1}}
\def\Z{{\mathbb Z}}
\def\half{\frac{1}{2}}
\def\VM{V^{\delta,\mu}}
\def\hwv{\ket{\delta,\mu}}
%\def\a#1{ a_{#1} }
\def\a#1{ a(#1) }
\def\rket#1{ | #1 ) }
\def\proof#1{{\bf Proof:} #1 $\blacksquare$ \medskip}
\def\id{\mbox{id}}
\def\pp{\phantom{\frac12}}
%
%\renewcommand{\thesection}{\Roman{section}.}
%\renewcommand{\theequation}{\thesection.\arabic{equation}}
%\renewcommand{\theequation}{\arabic{section}.\arabic{equation}}
%\renewcommand{\thefigure}{\thesection.\arabic{figure}}
%
\newtheorem{lemma}{Lemma}%[section]
\newtheorem{prop}[lemma]{Proposition}
%\newtheorem{thm}[lemma]{Theorem}
\newtheorem{thm}{Theorem}
\newtheorem{cor}[lemma]{Corollary}
\newtheorem{defo}{Definition}
%
%
%
\begin{document}

\begin{center}
{\large\bf Raising and lowering operators of $gl(m|n)$}\\
~~\\

%{\large Jason L. Werry, Mark D. Gould and Phillip S. Isaac }\\
%~~\\

School of Mathematics and Physics, The University of Queensland, St Lucia QLD 4072, Australia.
\end{center}

\begin{abstract}
We obtain the raising and lowering operators for $gl(m|n)$. 
\end{abstract}

% \begin{flushright}
% May 9, 2012
% \end{flushright}
% 
% \begin{tableofcontents}
% \end{tableofcontents}

\section{Introduction}
 
In this paper we generalize the results of \cite{Gould1978} and \cite{Gould1980} to obtain the raising and lowering operators for $gl(m|n)$. The non-super case was done by \cite{Bin1983} via the ... method.

The presentation of these results are as follows... 

%Section \ref{prelim} provides a brief review of .... The details of the characteristic identity are used to ..... in Section \ref{charident}, followed by a construction of the explicit raising and lowering formulae in Section \ref{Casimir}. Finally, Section \ref{resultsum} concludes with a discussion of the main results.  

%%%%%%%%%%%%%%%%%%%%%%%%%%%%%%%%%%%%%%%%%%%%%%%%%%%%%%%%%%%%%%%%%%%%%%%%%%%%%%%%%%%%%%%
%%%%%%%%%%%%%%%%%%%%%%%%%%%%%%%%%%%%%%%%%%%%%%%%%%%%%%%%%%%%%%%%%%%%%%%%%%%%%%%%%%%%%%%

\section{Preliminaries}
\label{prelim}

The $gl(m|n)$ generators $E_{pq}$ satisfy the graded commutation relations
$$
[E_{pq},E_{rs}] = \delta_{qr}E_{ps} - (-1)^{[(p)+(q)][(r)+(s)]}\delta_{ps}E_{rq}.
$$
where the Cartan subalgebra is given by the set of mutually commuting generators $E_{aa}$ 
whose eigenvalues label the weights occurring in a given $gl(m|n)$ module. 


%%%%%%%%%%%%%%%%%%%%%%%%%%%%%%%%%%%%%%%%%%%%%%%%%%%%%%%%%%%%%%%%%%%%%%%%%%%%%%%%%%%%%%%
%%%%%%%%%%%%%%%%%%%%%%%%%%%%%%%%%%%%%%%%%%%%%%%%%%%%%%%%%%%%%%%%%%%%%%%%%%%%%%%%%%%%%%%
%%%%%%%%%%%%%%%%%%%%%%%%%%%%%%%%%%%%%%%%%%%%%%%%%%%%%%%%%%%%%%%%%%%%%%%%%%%%%%%%%%%%%%%
%%%%%%%%%%%%%%%%%%%%%%%%%%%%%%%%%%%%%%%%%%%%%%%%%%%%%%%%%%%%%%%%%%%%%%%%%%%%%%%%%%%%%%%
\newpage
\section{The characteristic identity} 
\label{charident}

We now define the $gl(m|n)$ vector matrix \cite{GIW1}
\begin{align*}
	{\cal A}^{p}_{\ q} &= - \sum_{r,s} (-1)^{(s)+((r) + (s))(q)} \tilde{\pi}
(E_{rs})_{p q} E_{sr}\nn\\
	&= (-1)^{(p)}E_{p q}.\label{equdefx}
\end{align*}
which when acting on the irreducible
$gl(m|n)$ module $V(\Lambda)$ satisfies the characteristic identity
\begin{align*}
	\prod^m_{i=1} ({\cal A} - \alpha_i) \prod^n_{\mu=1} ({\cal A} - \alpha_\mu) = 0 
\end{align*}
where the characteristic roots are given by
\begin{align*}
\alpha_i &= \Lambda_i + m - n - i,\ \ 1\leq i\leq m \\
\alpha_\mu &= \mu-\Lambda_\mu - n,\ \ 1\leq \mu \leq n.
\end{align*}

Polynomials in ${\cal A}$ are defined recursively from the relations
\begin{align*}
({\cal A}^{k+1})^p_{\ q} &= \sum_r({\cal A}^k)^p_{\ r}\  {\cal A}^r_{\ q} = \sum_r
{\cal A}^p_{\ r}\ ({\cal A}^k)^r_{\ q} .\nn
\end{align*}



%%%%%%%%%%%%%%%%%%%%%%%%%%%%%%%%%%%%%%%%%%%%%%%%%%%%%%%%%%%%%%%%%%%%%%%%%%%%%%%%%%%%%%%
%%%%%%%%%%%%%%%%%%%%%%%%%%%%%%%%%%%%%%%%%%%%%%%%%%%%%%%%%%%%%%%%%%%%%%%%%%%%%%%%%%%%%%%
%%%%%%%%%%%%%%%%%%%%%%%%%%%%%%%%%%%%%%%%%%%%%%%%%%%%%%%%%%%%%%%%%%%%%%%%%%%%%%%%%%%%%%%
%%%%%%%%%%%%%%%%%%%%%%%%%%%%%%%%%%%%%%%%%%%%%%%%%%%%%%%%%%%%%%%%%%%%%%%%%%%%%%%%%%%%%%%

\section{Casimir invariants}
\label{Casimir}

Powers of these matrices are defined recursively as
\begin{align*}
\left({\cal A}^k\right)^p_q = \sum_{r=1}^{m+n} {\cal A}^p_r
\left({\cal A}^{k-1}\right)^r_q, ~~\left( \left({\cal A}^0 \right)^p_q \equiv
\delta_{p q} \right), ~~ 1\leq p,q\leq m+n.  
\end{align*}

By using induction and the $gl(m|n)$ commutation relations we obtain (see Appendix C in \cite{GIW1})
\begin{align*}
\left[ E^p_q , \left({\cal A}^k\right)^r_s \right] = (-1)^{(p) + (q)}
\delta^r_q \left({\cal A}^k\right)^p_s - (-1)^{((p) + (q))((r) +
(s))} \delta^p_s \left({\cal A}^k\right)^r_q.
\end{align*}
A special case of the above identity is then
\begin{align*}
\left[ E^p_q , \left({\cal A}^k\right)^q_p \right] = (-1)^{(p) + (q)}
 \left({\cal A}^k\right)^p_p - (-1)^{(p) + (q)}  \left({\cal A}^k\right)^q_q
\end{align*}
or equivalently
\begin{align*}
\left[ {\cal A}^p_q , \left({\cal A}^k\right)^{q}_{p} \right] = (-1)^{(p)} \left[
E^p_q , \left({\cal A}^k\right)^{q}_{p} \right] = (-1)^{(q)}
 \left({\cal A}^k\right)^p_p - (-1)^{(q)}  \left({\cal A}^k\right)^q_q.
\end{align*}
Also note that via recursive use of the commutation relations we have the 
following proposition (\textbf{which I need to check for }$gl(m|n)$)
\begin{prop} \label{PolyVanish}
Suppose  $h(x)$ is a polynomial over the field $F$. Then
$$
h({\cal A})^i_j v_0 = 0 \hbox{~for~} i<j
$$
\end{prop}
where $v_0$ is the highest weight of a finite dimensional irrep of $gl(m|n)$.

We now introduce the following operators 
$$
p_i(k) = \prod^{m+n}_{l=i} ({\cal A} - \alpha_l)^k_k,
$$
$$
p_{i,j}(k) = \prod^{m+n}_{l=i,l \neq j} ({\cal A} - \alpha_l)^k_k
$$
from which the following identities may easily be obtained
\begin{align*}
p_0(k) &= 0 \hbox{~~immediately from the characteristic identity}, \\
p_i(k) &= 0 \hbox{~~for~} k \geq i, \\
p_{i,j}(k) &= p_i(k) \hbox{~~for~} j < i.
\end{align*}
When acting on a maximal weight vector $v_0$ we let $\rho_i(k)$ and $\rho_{i,j}(k)$ denote
the eigenvalues of the operators $p_i(k)$ and $p_{i,j}(k)$ respectively.
\newpage
When $p_i(k)$ is acting on a $gl(m|n)$ module with highest weight $\Lambda$, we have
\begin{align}
p_i(k) &= \left[ ({\cal A} - \alpha_i) \prod_{j=i+1}^{m+n} ({\cal A} - \alpha_j) \right]^k_k \nn\\
&= \left[ {\cal A} \prod_{j=i+1}^{m+n} ({\cal A} - \alpha_j) \right]^k_k - \alpha_i \left[ \prod_{j=i+1}^{m+n} ({\cal A} - \alpha_j) \right]^k_k \nn\\
&= {\cal A}^k_l \left[ \prod_{j=i+1}^{m+n} ({\cal A} - \alpha_j) \right]^l_k - \alpha_i p_{i+1}(k),~~k \leq l \leq m+n \hbox{~from Proposition \ref{PolyVanish}}\nn\\
&= {\cal A}^k_l \left[ \prod_{j=i+1}^{m+n} ({\cal A} - \alpha_j) \right]^l_k + \Lambda_k p_{i+1}(k) - \alpha_i p_{i+1}(k),~~k+1 \leq l \leq m+n \nn\\
&= \left[{\cal A}^k_l , \left[\prod_{j=i+1}^{m+n} ({\cal A} - \alpha_j) \right]^l_k \right] 
+ \left[\prod_{j=i+1}^{m+n} ({\cal A} - \alpha_j) \right]^l_k {\cal A}^k_l 
+ (\Lambda_k - \alpha_i) p_{i+1}(k),~~k+1 \leq l \leq m+n \nn \\
&= \sum_{l=k+1}^{m+n} \left[{\cal A}^k_l , \left[\prod_{j=i+1}^{m+n} ({\cal A} - \alpha_j) \right]^l_k \right] 
+ (\Lambda_k - \alpha_i) p_{i+1}(k),  \hbox{~from Proposition \ref{PolyVanish}} \nn\\
&= \sum_{l=k+1}^{m+n} (-1)^{(l)} \left[\prod_{j=i+1}^{m+n} ({\cal A} - \alpha_j) \right]^k_k - \sum_{l=k+1}^{m+n} (-1)^{(l)} \left[\prod_{j=i+1}^{m+n} ({\cal A} - \alpha_j) \right]^l_l + (\Lambda_k - \alpha_i) p_{i+1}(k) \nn \\
&= \sum_{l=k+1}^{m+n} (-1)^{(l)} p_{i+1}(k) - \sum_{l=k+1}^{m+n} (-1)^{(l)}  p_{i+1}(l) + (\Lambda_k - \alpha_i) p_{i+1}(k) \nn .
\end{align}
Now, for $k$ even we have
\begin{align*}
\Lambda_k + \sum_{l=k+1}^{m+n} (-1)^{(l)} &= m - k - n + \Lambda_k\\
 &= \alpha_k, ~(k) = 0
\end{align*}
while for $k$ odd
\begin{align*}
\Lambda_k + \sum_{l=k+1}^{m+n} (-1)^{(l)} &= k - m - n + \Lambda_k \\
&= \alpha_k, ~(k) = 1
\end{align*}
therefore, by letting $\rho_i(k)$ denote the eigenvalue of the operator $p_i(k)$ we have 
\begin{align}
\rho_i(k) &= (\alpha_k - \alpha_i)\rho_{i+1}(k) - \rho_{i+1}(k+1) - ... - \rho_{i+1}(m) + \rho_{i+1}(m+1) +...+ \rho_{i+1}(m+n) \nn.
\end{align}
Note, this relation is the super-analogue of the $gl(n)$ recursive relation given in \cite{Gould1978} (see Appendix 2).
Similarly, for $i<j-1$ we find that
\begin{align}
\rho_{i,j}(k) &= (\alpha_k - \alpha_i)\rho_{i+1,j}(k) - \rho_{i+1,j}(k+1) - ... - \rho_{i+1,j}(m) + \rho_{i+1,j}(m+1) +...+ \rho_{i+1,j}(m+n) \nn.
\end{align}
We then obtain (Appendix 1)
\begin{align}
\rho_{1,r}(k) = \prod_{l < k} (\alpha_r - \alpha_l) \prod_{l > k} (\alpha_r - \alpha_l - (-1)^{(l)}),~~k \leq r
\label{MainIdentity}
\end{align}
which is the super-analogue of Eqn (12) in \cite{Gould1978}.

From the definitions we have
$$
p_{1,r}(k) = \prod^{m+n}_{l=1,l \neq r} ({\cal A} - \alpha_l)^k_k
$$
and
$$
f_r = \prod_{l=1,l \neq r}^{m+n} \left( \frac{{\cal A} - \alpha_l}{\alpha_r - \alpha_l}  \right)
$$
resulting in the relation
$$
(f_r)^k_k \nu_0 = \prod_{l \neq r}(\alpha_r - \alpha_l)^{-1} \rho_{1,r}(k) \nu_0.
$$
\textbf{What about atypicality causing $\alpha_r = \alpha_l$ in all of the above?}
From equation (\ref{MainIdentity}) we observe that
\begin{align*}
\prod_{l \neq r}(\alpha_r - \alpha_l)^{-1} \rho_{1,r}(k) &= \prod_{l \neq r}(\alpha_r - \alpha_l)^{-1} \prod_{l < k} (\alpha_r - \alpha_l) \prod_{l > k} (\alpha_r - \alpha_l - (-1)^{(l)}),~~k \leq r \\
&= \prod_{l \geq k,l \neq r}(\alpha_r - \alpha_l)^{-1} \prod_{l > k} (\alpha_r - \alpha_l - (-1)^{(l)}),~~k \leq r \\
&= \underbrace{(\alpha_r - \alpha_k)^{-1}}_{?} 
\prod_{l > k} \left( \frac{\alpha_r - \alpha_l - (-1)^{(l)} }{\alpha_r - \alpha_l + \delta_{rl} } \right),~~k \leq r 
\end{align*}
to give
\begin{align}
(f_r)^k_k \nu_0 = \prod_{l > k} \left( \frac{\alpha_r - \alpha_l - (-1)^{(l)} }{\alpha_r - \alpha_l + \delta_{rl} } \right),~~k \leq r .
\label{Projection_kk}
\end{align}
By summing (how? identity?) the above equation over $k$ we obtain
$$
\chi_\mu(\hbox{str} f_k) = \prod_{l \neq k}^{m+n} \left( \frac{\alpha_k - \alpha_l - (-1)^{(l)} }{\alpha_k - \alpha_l} \right)
$$
which agrees with the supertrace formula, previously derived by Jarvis and Green, J. Math. Phys. 20, 2115 (1979). and with the supertrace formula obtained in Math. Phys. 54, 013505 (2013).

\section{Raising and lowering operators of $gl(m|n)$}

The lowering generator is given by
$$
N^r_{m+n}  = \left((f_r)^r_r \delta_r \right)^{1/2}.
$$
We therefore calculate
\begin{align*}
\left(N^r_{m+n}\right)^2 &=  \prod_{l > r}^{m+n} \left( \frac{\alpha_l - \alpha_r + (-1)^{(l)} }{\alpha_l - \alpha_r  } \right) \prod_{l\in I, l\neq r} (\alpha_l - \alpha_r +
(-1)^{(l)})^{-1}\prod_{l\in \tilde{I}} (\beta_l-\alpha_r), \ \ r\in I' .
%\nn\\
%&= \prod_{l \in \tilde{I}} ( \beta_l - \alpha_r   )  
%\prod_{l \in I, l< r} ( \alpha_l - \alpha_r + (-1)^{(l)} )^{-1} 
%\prod_{l \in I, l > r} ( \alpha_l - \alpha_r )^{-1}  \ \ r\in I'
\end{align*}

Now since $I_0 \cup \bar{I}_0 = \{1,...,m\}$ and $I_0 \cap \bar{I}_0 = \emptyset$ then for a general function of the roots $g$
$$
\prod_{l > r}^m g(l) = \prod_{l \in I_0, l > r} g(l) \prod_{l \in \bar{I}_0, l > r} g(l),
$$
so that
\begin{align*}
\prod_{l > r}^{m+n} g(l) &= \prod_{l \in I_0, l > r} g(l) \prod_{l \in \bar{I}_0, l > r} g(l) \prod_{l = n + 1, l > r}^{m+n} g(l) \\
&=   \prod_{l \in I, l > r} g(l) \prod_{l \in \bar{I}_0, l > r} g(l) .
\end{align*}
We then have
\[
\fbox{ 
\addtolength{\linewidth}{-2\fboxsep}%
\addtolength{\linewidth}{-2\fboxrule}%
\begin{minipage}{\linewidth}
\begin{align*}
(N^r_{m+n})^2 &=  
\prod_{l \in I, l< r} ( \alpha_l - \alpha_r + (-1)^{(l)} )^{-1} 
\prod_{l \in I, l > r} ( \alpha_l - \alpha_r )^{-1} 
\prod_{l \in \bar{I}_0, l > r} \left( \frac{\alpha_l - \alpha_r + 1 }{\alpha_l - \alpha_r  } \right)
\prod_{l \in \tilde{I}} ( \beta_l - \alpha_r   ) 
 \ \ r\in I'
\end{align*}
\end{minipage}
}
\]
which for $\bar{I}_0 = \emptyset$ and $(l) = 0$ is the same as the classical case
\begin{align*}
(N^r_{m+n})^2 &=  
\prod_{l = 1, l < r}^m ( \alpha_l - \alpha_r + 1 )^{-1} 
\prod_{l = 1, l > r}^m ( \alpha_l - \alpha_r )^{-1} 
\prod_{l = 1}^{m+1} ( \beta_l - \alpha_r   ) .
\end{align*}
Furthermore, from the index set free expression for $\delta_r$ 
\begin{align*}
\delta_r &= (-1)^{m + n} \prod_{l =1,l \neq r}^m \left( \frac{\alpha_l - \alpha_r}{\alpha_r - \beta_l - 1} \right) \prod^{m+n}_{l=m+1,l \neq r} (\alpha_r - \alpha_l + 1)^{-1} \prod^{m+n+1}_{l=m+1} (\beta_l - \alpha_r) \\
&= - \prod_{l =1,l \neq r}^m \left( \frac{\alpha_l - \alpha_r}{\beta_l - \alpha_r + 1} \right) \prod^{m+n}_{l=m+1,l \neq r} (\alpha_l - \alpha_r - 1)^{-1} \prod^{m+n+1}_{l=m+1} (\beta_l - \alpha_r)
,~r\in I'
\end{align*}
we also have an equivalent index-set free expression for $r\in I'$
\[
\fbox{ 
\addtolength{\linewidth}{-2\fboxsep}%
\addtolength{\linewidth}{-2\fboxrule}%
\begin{minipage}{\linewidth}
\begin{align*}
\left(N^r_{m+n}\right)^2 &= - \prod_{l > r}^{m+n} \left( \frac{\alpha_l - \alpha_r + (-1)^{(l)} }{\alpha_l - \alpha_r  } \right)
\prod_{l =1,l \neq r}^m \left( \frac{\alpha_l - \alpha_r}{\beta_l - \alpha_r + 1} \right) 
\prod^{m+n}_{l=m+1,l \neq r} (\alpha_l - \alpha_r - 1)^{-1} 
\prod^{m+n+1}_{l=m+1} (\beta_l - \alpha_r).
\end{align*}
\end{minipage}
}
\]
Now, for the (even semi-minimal) case $\alpha_l = \beta_l, \forall 1 \leq l \leq m$, we have
$$
\prod_{l =1,l \neq r}^m \left( \frac{\alpha_l - \alpha_r}{\beta_l - \alpha_r + 1} \right) = \prod_{l =1,l \neq r}^m \left( \frac{\beta_l - \alpha_r}{\alpha_l - \alpha_r + 1}, \right)
$$
so 
\begin{align*}
\left(N^r_{m+n}\right)^2 &= - \prod_{l > r}^{m+n} \left( \frac{\alpha_l - \alpha_r + (-1)^{(l)} }{\alpha_l - \alpha_r  } \right)
\prod_{l =1,l \neq r}^{m+n} \left( \frac{\beta_l - \alpha_r}{\alpha_l - \alpha_r + (-1)^{(l)}} \right) \times (\beta_{m+n+1} - \alpha_r)\\
&= - \prod_{l < r}^{m+n} \left( \frac{\beta_l - \alpha_r}{\alpha_l - \alpha_r + (-1)^{(l)}} \right) 
\prod_{l > r}^{m+n} \left( \frac{\beta_l - \alpha_r  }{\alpha_l - \alpha_r  } \right) \times (\beta_{m+n+1} - \alpha_r) \\
&= - \prod_{l \neq r}^{m+n+1} ( \beta_l - \alpha_r) \prod_{l < r}^{m+n} (\alpha_l - \alpha_r + (-1)^{(l)} )^{-1} \prod_{l > r}^{m+n} (\alpha_l - \alpha_r)^{-1}
\end{align*}

Similarly, for the (even semi-maximal) case $\alpha_l = \beta_l + 1, \forall 1 \leq l \leq m$, then
$$
\prod_{l =1,l \neq r}^m \left( \frac{\alpha_l - \alpha_r}{\beta_l - \alpha_r + 1} \right) = 1
$$
which gives
\begin{align*}
\left(N^r_{m+n}\right)^2 &= - \prod_{l > r}^{m+n} \left( \frac{\alpha_l - \alpha_r + (-1)^{(l)} }{\alpha_l - \alpha_r  } \right) 
\prod^{m+n}_{l=m+1,l \neq r} (\alpha_l - \alpha_r - 1)^{-1} 
\prod^{m+n+1}_{l=m+1} (\beta_l - \alpha_r).
\end{align*}
Setting $r \leq m+1$ we have
\begin{align*}
\left(N^r_{m+n}\right)^2 &= - \prod_{l > r}^{m} \left( \frac{\alpha_l - \alpha_r + 1 }{\alpha_l - \alpha_r  } \right) 
\prod^{m+n}_{l=m+1} (\alpha_l - \alpha_r)^{-1} 
\prod^{m+n+1}_{l=m+1} (\beta_l - \alpha_r).
\end{align*}
%which has various equivalent expressions...
%\begin{align*}
%(N^r_{m+n})^2 &= \prod_{l \in I, l< r} ( \alpha_l - \alpha_r + (-1)^{(l)} )^{-1} 
%\prod_{l \in I_0, l > r} ( \alpha_l - \alpha_r )^{-1}  \\
%&~\times \prod_{l \in \bar{I}_0, l > r} \left( \frac{\alpha_l - \alpha_r + 1 }{\alpha_l - \alpha_r  } \right) 
%\prod_{l \in \tilde{I}} ( \beta_l - \alpha_r   ) \prod_{l = m+1, l > r}^{m+n} ( \alpha_l - \alpha_r )^{-1} \\
%&=   \prod_{l \in I, l< r} ( \alpha_l - \alpha_r + (-1)^{(l)} )^{-1} \prod_{l \in \bar{I}_0, l > r} ( \alpha_l - \alpha_r + 1 ) \\
%&~\times  
%\prod_{l \in \tilde{I}} ( \beta_l - \alpha_r   ) \prod_{l = 1, l > r}^{m+n} ( \alpha_l - \alpha_r )^{-1} 
%\end{align*}

%which has various equivalent expressions...
%\begin{align*}
%(N^r_{m+n})^2 &=  
%\prod_{l \in I, l< r} ( \alpha_l - \alpha_r + (-1)^{(l)} )^{-1} 
%\prod_{l \in I, l > r} ( \alpha_l - \alpha_r )^{-1} 
%\prod_{l \in \bar{I}_0, l > r} \left( \frac{\alpha_l - \alpha_r + (-1)^{(l)} }{\alpha_l - \alpha_r  } \right) \\
%&~\times \prod_{l \in I_0} ( \alpha_l - \alpha_r   ) \prod_{l = m+1}^{m+n+1} ( \beta_l - \alpha_r   ) \\
%&=  
%\prod_{l \in I, l< r} ( \alpha_l - \alpha_r + (-1)^{(l)} )^{-1} 
%\prod_{l = m + 1, l > r}^{m+n} ( \alpha_l - \alpha_r )^{-1} 
%\prod_{l \in \bar{I}_0, l > r} \left( \frac{\alpha_l - \alpha_r + (-1)^{(l)} }{\alpha_l - \alpha_r  } \right) \\
%&~\times \prod_{l \in I_0,l < r} ( \alpha_l - \alpha_r   ) \prod_{l = m+1}^{m+n+1} ( \beta_l - \alpha_r   )\\ 
%&=  
%\prod_{l \in I, l< r} ( \alpha_l - \alpha_r + (-1)^{(l)} )^{-1} 
%\prod_{l \in I', l > r}^{m+n} ( \alpha_l - \alpha_r )^{-1} 
%\prod_{l \in \bar{I}_0, l > r} ( \alpha_l - \alpha_r + 1) \\
%&~\times \prod_{l \in I_0,l < r} ( \alpha_l - \alpha_r   ) \prod_{l = m+1}^{m+n+1} ( \beta_l - \alpha_r   )
 %\ \ r\in I'
%\end{align*}


%~~\\
%~~\\
%~~\\
%\begin{align*}
%&= (\beta_{m+n+1}-\alpha_r) (\beta_r - \alpha_r) 
%\prod_{l\in I_0, l< r} \frac{ \beta_l-\alpha_r   } 
%{ \alpha_l - \alpha_r + (-1)^{(l)} } 
%\prod_{l\in I_1, l< r} \frac{ \beta_l-\alpha_r   } 
%{ \alpha_l - \alpha_r + (-1)^{(l)} } \\
%&~\times
%\prod_{l\in I_0, l > r} \frac{ \beta_l-\alpha_r   }
%{ \alpha_l - \alpha_r       }
%\prod_{l\in I_1, l > r} \frac{ \beta_l-\alpha_r   } 
%{ \alpha_l - \alpha_r       } \\
%&= (\beta_{m+n+1}-\alpha_r) (\beta_r - \alpha_r) 
%\prod_{l\in I_0, l< r} \frac{ \alpha_l-\alpha_r   } 
%{ \alpha_l - \alpha_r + 1 } 
%\prod_{l\in I_1, l< r} \frac{ \beta_l-\alpha_r   } 
%{ \alpha_l - \alpha_r  - 1 } \\
%&~\times \prod_{l\in I_1, l > r} \frac{ \beta_l-\alpha_r   } 
%{ \alpha_l - \alpha_r       }
%\end{align*}

%%%%%%%%%%%%%%%%%%%%%%%%%%%%%%%%%
%%%%%%%%%%%%%%%%%%%%%%%%%%%%%%%%%
\newpage
\section*{Appendix 1 - $gl(m|n)$ examples}
\begin{center}
{\large\bf $gl(m|n),~m=2,n=2:~\rho_{1,4}(2)$} - atypicality? Type 1 vs Type 2?
\end{center}
The characteristic matrix is $A_{pq} = (-1)^{(p)} E_{pq}$.
The roots are
$$
\alpha_1 = \Lambda_1 - 1,~ \alpha_2 = \Lambda_2 - 2,~ \alpha_3 = -\Lambda_3 - 1,~\alpha_4 = -\Lambda_4 
$$
We have the difference equation
$$
\rho_{i,j}(k) = (\alpha_k - \alpha_i) \rho_{i+1,j}(k) - \rho_{i+1,j}(k+1) - ... + \rho_{i+1,j}(m+1) + ... + \rho_{i+1,j}(m+n),
$$
For $m=n=2$ we then have:
\begin{align*}
\rho_{3,4}(4) &= \prod_{l=3,l \neq 4}^4(A - \alpha_l)^4_4 \\
&= -\Lambda_4 - \alpha_3 \\
&= \alpha_4 - \alpha_3 
\end{align*}
\begin{align*}
\rho_{2,4}(4) &= (\alpha_4 - \alpha_2) \rho_{3,4}(4) \\
&= (\alpha_4 - \alpha_3)(\alpha_4 - \alpha_2)
\end{align*}
\begin{align*}
\rho_{3,4}(3) &=  \prod_{l=3,l \neq 4}^4(A - \alpha_l)^3_3\\
&=  (A - \alpha_3)^3_3 \\
&= -\Lambda_3 - \alpha_3 \\
&= -\Lambda_3 - (-\Lambda_3 - 1) \\
&= 1
\end{align*}
\begin{align*}
\rho_{2,4}(3) &= (\alpha_3 - \alpha_2) \rho_{3,4}(3) + \rho_{3,4}(4) \\
&=^? (\alpha_3 - \alpha_2) + (\alpha_4 - \alpha_3) \\
&= (\alpha_4 - \alpha_2)
\end{align*}
\begin{align*}
\rho_{2,4}(2) &=  \rho_{3,4}(3) + \rho_{3,4}(4)\\
&= 1 + (\alpha_4 - \alpha_3)
\end{align*}
\begin{align*}
\rho_{1,4}(2) &= (\alpha_2 - \alpha_1) \rho_{2,4}(2) + \rho_{2,4}(3) + \rho_{2,4}(4) \\
&= (\alpha_2 - \alpha_1) (1 + \alpha_4 - \alpha_3) + (\alpha_4 - \alpha_2) + (\alpha_4 - \alpha_3)(\alpha_4 - \alpha_2) \\
&= (\alpha_2 - \alpha_1)  + (\alpha_2 - \alpha_1) (\alpha_4 - \alpha_3)  + (\alpha_4 - \alpha_2) + (\alpha_4 - \alpha_3)(\alpha_4 - \alpha_2) \\
&= (\alpha_4 - \alpha_1)  + (\alpha_4 - \alpha_3)( \alpha_2 - \alpha_1 + \alpha_4 - \alpha_2 ) \\
&= (\alpha_4 - \alpha_1)  + (\alpha_4 - \alpha_3)(\alpha_4 - \alpha_1  ) \\
&= (\alpha_4 - \alpha_1) ( 1 + \alpha_4 - \alpha_3) \\
&= (\alpha_4 - \alpha_1) ( \alpha_4 - \alpha_3 + 1)
\end{align*}
so we conjecture that
$$
\rho_{1,r}(k) = \prod_{l < k} (\alpha_r - \alpha_l) \prod_{l > k} (\alpha_r - \alpha_l - (-1)^{(l)}),~~k \leq r
$$
\newpage
We now check the grading for $l=3$ by considering $gl(3|1)$ so that $m=3,n=1$.

The characteristic matrix is $A_{pq} = (-1)^{(p)} E_{pq}$.
The roots are
$$
\alpha_1 = \Lambda_1 + 1,~ \alpha_2 = \Lambda_2 ,~ \alpha_3 = \Lambda_3 - 1,~\alpha_4 = -\Lambda_4 
$$
We have the difference equation
$$
\rho_{i,j}(k) = (\alpha_k - \alpha_i) \rho_{i+1,j}(k) - \rho_{i+1,j}(k+1) - ... - \rho_{i+1,j}(m+n-1) + \rho_{i+1,j}(m+n),
$$
For $m=3,n=1$ we then have:
\begin{align*}
\rho_{3,4}(4) &= \prod_{l=3,l \neq 4}^4(A - \alpha_l)^4_4 \\
&= -\Lambda_4 - \alpha_3 \\
&= \alpha_4 - \alpha_3 
\end{align*}
\begin{align*}
\rho_{2,4}(4) &= (\alpha_4 - \alpha_2) \rho_{3,4}(4) \\
&= (\alpha_4 - \alpha_3)(\alpha_4 - \alpha_2)
\end{align*}
\begin{align*}
\rho_{3,4}(3) &=  \prod_{l=3,l \neq 4}^4(A - \alpha_l)^3_3\\
&=  (A - \alpha_3)^3_3 \\
&= \Lambda_3 - \alpha_3 \\
&= \Lambda_3 - (\Lambda_3 - 1) \\
&= 1
\end{align*}
\begin{align*}
\rho_{2,4}(3) &= (\alpha_3 - \alpha_2) \rho_{3,4}(3) + \rho_{3,4}(4) \\
&=^? (\alpha_3 - \alpha_2) + (\alpha_4 - \alpha_3) \\
&= (\alpha_4 - \alpha_2)
\end{align*}
\begin{align*}
\rho_{2,4}(2) &=  -\rho_{3,4}(3) + \rho_{3,4}(4)\\
&= -1 + (\alpha_4 - \alpha_3)
\end{align*}
\begin{align*}
\rho_{1,4}(2) &= (\alpha_2 - \alpha_1) \rho_{2,4}(2) - \rho_{2,4}(3) + \rho_{2,4}(4) \\
&= (\alpha_2 - \alpha_1) (-1 + \alpha_4 - \alpha_3) - (\alpha_4 - \alpha_2) + (\alpha_4 - \alpha_3)(\alpha_4 - \alpha_2) \\
&= -(\alpha_2 - \alpha_1)  + (\alpha_2 - \alpha_1) (\alpha_4 - \alpha_3) - (\alpha_4 - \alpha_2) + (\alpha_4 - \alpha_3)(\alpha_4 - \alpha_2) \\
&= (\alpha_1 - \alpha_4)  + (\alpha_4 - \alpha_3)( \alpha_2 - \alpha_1 + \alpha_4 - \alpha_2 ) \\
&= (\alpha_1 - \alpha_4)  + (\alpha_4 - \alpha_3)(\alpha_4 - \alpha_1  ) \\
&= (\alpha_4 - \alpha_1) ( -1 + \alpha_4 - \alpha_3) \\
&= (\alpha_4 - \alpha_1) ( \alpha_4 - \alpha_3 - 1)
\end{align*}
which supports the conjecture.

\newpage
We now check the grading for $l=2$ by considering $gl(1|3)$ so that $m=1,n=3$....

\newpage
\section*{Appendix 2 - The $gl(n)$ recursive relation}
Here we derive the 'difference equation' given on page 407 of \cite{Gould1978}.
Let 
$$
p_i(k) = \left[ \prod^{n}_{l=i} ({\cal A} - \alpha_l) \right]^k_k,~~Q= \prod^{n}_{l=i+1} ({\cal A} - \alpha_l)
$$
then
\begin{align}
p_i(k) &= \left[ ({\cal A} - \alpha_i) Q \right]^k_k \nn\\
&= \left[ {\cal A} Q \right]^k_k - \left[ \alpha_i Q \right]^k_k \nn\\
&= {\cal A}^k_l \left[ Q \right]^l_k - \alpha_i p_{i+1}(k),~~l \geq k \nn\\
&= {\cal A}^k_l \left[ Q \right]^l_k + \alpha_k p_{i+1}(k) - \alpha_i p_{i+1}(k),~~l \geq k+1 \nn\\
&= \left[{\cal A}^k_l , \left[Q \right]^l_k \right] 
+ \left[Q \right]^l_k {\cal A}^k_l 
+ (\alpha_k - \alpha_i) p_{i+1}(k),~~l \geq k+1 \nn \\
&= \sum_{l=k+1}^{n} \left[{\cal A}^k_l , \left[Q \right]^l_k \right] 
+ (\alpha_k - \alpha_i) p_{i+1}(k), \nn \hbox{~from Proposition \ref{PolyVanish}} \nn\\
&= \sum_{l=k+1}^{n} \left[Q \right]^k_k - \sum_{l=k+1}^{n} \left[Q \right]^l_l + (\alpha_k - \alpha_i) p_{i+1}(k) \nn \\
&= \sum_{l=k+1}^{n}  p_{i+1}(k) - \sum_{l=k+1}^{n} p_{i+1}(l) + (\alpha_k - \alpha_i) p_{i+1}(k) \nn \\
&= \sum_{l=k+1}^{n}  p_{i+1}(k)  + (\alpha_k - \alpha_i)p_{i+1}(k) - p_{i+1}(k+1) - ... - p_{i+1}(n) .
\end{align}

\newpage
\section*{Appendix 3 - $gl(n)$ examples}

\begin{center}
{$gl(n),~n=4:~\rho_{1,4}(3)$}
\end{center}
We have the operators
$$
p_i(k) = \prod_{j=i}^n(A - \alpha_j)^k_k,
$$
$$
p_{i,j}(k) = \prod_{l=i,l \neq j}^n(A - \alpha_l)^k_k,
$$
and the difference equations
$$
\rho_i(k) = (\alpha_k - \alpha_i) \rho_{i+1}(k) - \rho_{i+1}(k+1) - ... - \rho_{i+1}(n),
$$
$$
\rho_{i,j}(k) = (\alpha_k - \alpha_i) \rho_{i+1,j}(k) - \rho_{i+1,j}(k+1) - ... - \rho_{i+1,j}(n),
$$
while we also note that
$$
\rho_{i,j} = \rho_i \hbox{~for~} j < i,~~~\rho_{j,j}(k) = \rho_{j+1}(k),
$$
$$
\rho_i(k) = 0 \hbox{~for~} k \geq i,~~~\rho_{i,j}(k) = 0 \hbox{~for~} k > j ~(\hbox{and here we must have~} i \leq j).
$$
For $n=4$ we then have:
\begin{align*}
\rho_{3,4}(4) &= (\alpha_4 - \alpha_3)
\end{align*}
\begin{align*}
\rho_{2,4}(4) &= (\alpha_4 - \alpha_3)(\alpha_4 - \alpha_2)
\end{align*}
\begin{align*}
\rho_{3,4}(3) &=  \prod_{l=3,l \neq 4}^4(A - \alpha_l)^3_3\\
&=  (A - \alpha_3)^3_3 \\
&= \Lambda_3 - \alpha_3 \\
&= \Lambda_3 - (\Lambda_3 + n - 3) \\
&= -1
\end{align*}
\begin{align*}
\rho_{2,4}(3) &= (\alpha_3 - \alpha_2) \rho_{3,4}(3) - \rho_{3,4}(4) \\
&= -(\alpha_3 - \alpha_2) - (\alpha_4 - \alpha_3) \\
&= (\alpha_2 - \alpha_4)
\end{align*}
\begin{align*}
\rho_{1,4}(3) &= (\alpha_3 - \alpha_1) \rho_{2,4}(3) - \rho_{2,4}(4) \\
&= (\alpha_3 - \alpha_1) (\alpha_2 - \alpha_4) - (\alpha_4 - \alpha_3)(\alpha_4 - \alpha_2)\\
&= (\alpha_1 - \alpha_3) (\alpha_4 - \alpha_2) - (\alpha_4 - \alpha_3)(\alpha_4 - \alpha_2) \\
&= (\alpha_4 - \alpha_2)(\alpha_1 - \alpha_3 - \alpha_4 + \alpha_3 ) \\
&= -(\alpha_4 - \alpha_2)(\alpha_4 - \alpha_1)
\end{align*}


\newpage
\begin{center}
{\large\bf $gl(n),~n=4:~\rho_{1,4}(2)$}
\end{center}
We have the operators
$$
p_i(k) = \prod_{j=i}^n(A - \alpha_j)^k_k,
$$
$$
p_{i,j}(k) = \prod_{l=i,l \neq j}^n(A - \alpha_l)^k_k,
$$
and the difference equations
$$
\rho_i(k) = (\alpha_k - \alpha_i) \rho_{i+1}(k) - \rho_{i+1}(k+1) - ... - \rho_{i+1}(n),
$$
$$
\rho_{i,j}(k) = (\alpha_k - \alpha_i) \rho_{i+1,j}(k) - \rho_{i+1,j}(k+1) - ... - \rho_{i+1,j}(n),
$$

For $n=4$ we then have:
\begin{align*}
\rho_{3,4}(4) &= (\alpha_4 - \alpha_3)
\end{align*}
\begin{align*}
\rho_{2,4}(4) &= (\alpha_4 - \alpha_3)(\alpha_4 - \alpha_2)
\end{align*}
\begin{align*}
\rho_{3,4}(3) &=  \prod_{l=3,l \neq 4}^4(A - \alpha_l)^3_3\\
&=  (A - \alpha_3)^3_3 \\
&= \Lambda_3 - \alpha_3 \\
&= \Lambda_3 - (\Lambda_3 + n - 3) \\
&= -1
\end{align*}
\begin{align*}
\rho_{2,4}(3) &= (\alpha_3 - \alpha_2) \rho_{3,4}(3) - \rho_{3,4}(4) \\
&= -(\alpha_3 - \alpha_2) - (\alpha_4 - \alpha_3) \\
&= (\alpha_2 - \alpha_4)
\end{align*}
\begin{align*}
\rho_{2,4}(2) &= - \rho_{3,4}(3) - \rho_{3,4}(4)\\
&= 1 - (\alpha_4 - \alpha_3)
\end{align*}
\begin{align*}
\rho_{1,4}(2) &= (\alpha_2 - \alpha_1) \rho_{2,4}(2) - \rho_{2,4}(3) - \rho_{2,4}(4) \\
&= (\alpha_2 - \alpha_1) (1 - \alpha_4 + \alpha_3) - (\alpha_2 - \alpha_4) - (\alpha_4 - \alpha_3)(\alpha_4 - \alpha_2) \\
&= (\alpha_2 - \alpha_1)  + (\alpha_2 - \alpha_1) (\alpha_3 - \alpha_4)  - (\alpha_2 - \alpha_4) - (\alpha_4 - \alpha_3)(\alpha_4 - \alpha_2) \\
&= (\alpha_4 - \alpha_1)  - (\alpha_4 - \alpha_3)( \alpha_2 - \alpha_1 + \alpha_4 - \alpha_2 ) \\
&= (\alpha_4 - \alpha_1)  - (\alpha_4 - \alpha_3)(\alpha_4 - \alpha_1  ) \\
&= (\alpha_4 - \alpha_1) ( 1 -\alpha_4 + \alpha_3) \\
&= -(\alpha_4 - \alpha_1) ( \alpha_4 - \alpha_3 - 1)
\end{align*}

%%%%%%%%%%%%%%%%%%%%%%%%%%%%%%%%%%%%%%%%%%%%%%%%%%%%%%%%%%%%%%%%%%%%%%%%%%%%%%%%%%%%%%%
%%%%%%%%%%%%%%%%%%%%%%%%%%%%%%%%%%%%%%%%%%%%%%%%%%%%%%%%%%%%%%%%%%%%%%%%%%%%%%%%%%%%%%%
%%%%%%%%%%%%%%%%%%%%%%%%%%%%%%%%%%%%%%%%%%%%%%%%%%%%%%%%%%%%%%%%%%%%%%%%%%%%%%%%%%%%%%%

%\section{Summary of main results} 
%%%%%%%%%%%%%%%%%%%%%%%%%%%%%%%%%%%%%%%%%%%%%%%%%%%%%%%%%%%%%%%%%%%%%%%%%%%%%%%%%%%%%%%%
%%%%%%%%%%%%%%%%%%%%%%%%%%%%%%%%%%%%%%%%%%%%%%%%%%%%%%%%%%%%%%%%%%%%%%%%%%%%%%%%%%%%%%%% 
%\label{resultsum}
%
%The raising and lowering operators are ...

%%%%%%%%%%%%%%%%%%%%%%%%%%%%%%%%%%%%%%%%%%%%%%%%%%%%%%%%%%%%%%%%%%%%%%%%%%%%%%%%%%%%%%%
%%%%%%%%%%%%%%%%%%%%%%%%%%%%%%%%%%%%%%%%%%%%%%%%%%%%%%%%%%%%%%%%%%%%%%%%%%%%%%%%%%%%%%%
%%%%%%%%%%%%%%%%%%%%%%%%%%%%%%%%%%%%%%%%%%%%%%%%%%%%%%%%%%%%%%%%%%%%%%%%%%%%%%%%%%%%%%%
%%%%%%%%%%%%%%%%%%%%%%%%%%%%%%%%%%%%%%%%%%%%%%%%%%%%%%%%%%%%%%%%%%%%%%%%%%%%%%%%%%%%%%%
%%%%%%%%%%%%%%%%%%%%%%%%%%%%%%%%%%%%%%%%%%%%%%%%%%%%%%%%%%%%%%%%%%%%%%%%%%%%%%%%%%%%%%%
%%%%%%%%%%%%%%%%%%%%%%%%%%%%%%%%%%%%%%%%%%%%%%%%%%%%%%%%%%%%%%%%%%%%%%%%%%%%%%%%%%%%%%%
%%%%%%%%%%%%%%%%%%%%%%%%%%%%%%%%%%%%%%%%%%%%%%%%%%%%%%%%%%%%%%%%%%%%%%%%%%%%%%%%%%%%%%%


%%%%%%%%%%%%%%%%%%%%%%%%%%%%%%%%%%%%%%%%%%%%%%%%%%%%%%%%%%%%%%%%%%%%%%%%%%%%%%
%
%  Acknowledgements
%
%%%%%%%%%%%%%%%%%%%%%%%%%%%%%%%%%%%%%%%%%%%%%%%%%%%%%%%%%%%%%%%%%%%%%%%%%%%%%%
%
%\section*{Acknowledgements}
%
%We would like to thank...
%
%%%%%%%%%%%%%%%%%%%%%%%%%%%%%%%%%%%%%%%%%%%%%%%%%%%%%%%%%%%%%%%%%%%%%%%%%%%%%%
%
%  References
%
%%%%%%%%%%%%%%%%%%%%%%%%%%%%%%%%%%%%%%%%%%%%%%%%%%%%%%%%%%%%%%%%%%%%%%%%%%%%%%
%
\newpage
\begin{thebibliography}{10}
\bibitem{Bin1983} A.M. Bincer, J. Math. Phys. {\bf 24} (1983) 2546.

%\bibitem{ZhaGou1990} M.D. Gould and R.B. Zhang, J. Math. Phys. {\bf 31} (1990) 2552.

%\bibitem{GouZha1990} M.D. Gould and R.B. Zhang, Lett. Math. Phys. {\bf 20} (1990) 221.

%\bibitem{SNR1977} M. Scheunert, W. Nahm and V. Rittenberg, J. Math. Phys. {\bf 18} (1977) 146. 
% 
\bibitem{GIW1} M.D. Gould, P.S. Isaac and J.L. Werry, J. Math. Phys. {\bf 54} (2013), 013505.
% 
%\bibitem{GIW2} M.D. Gould, P.S. Isaac and J.L. Werry, J. Math. Phys. {\bf 55} (2014), 011703.
% 
%\bibitem{Green1971} H.S. Green, J. Math. Phys. {\bf 12} (1971) 2106.
%  
%\bibitem{BraGre1971} A.J. Bracken and H.S. Green, J. Math. Phys. {\bf 12} (1971) 2099.
%  
% \bibitem{Green1975} H.S. Green, J. Austral. Math. Soc. Ser. B {\bf 19} (1975) 129.
% 
%\bibitem{OBCantCar1977} D.M. O'Brien, A. Cant and A.L. Carey, Ann. Inst. Henri 
%Poincar\'e, Section A: Physique th\'eorique {\bf 26} (1977) 405.
  
%\bibitem{Gould1985} M.D. Gould, J. Austral. Math. Soc. Ser. B {\bf 26}  (1985) 257.

%\bibitem{JarGre1979} P.D. Jarvis and H.S. Green, J. Math. Phys.  {\bf 20}  (1979) 2115.
 
%\bibitem{GreJar1983}  H.S. Green and P.D. Jarvis, J. Math. Phys.  {\bf 24}  (1983) 1681.
  
%\bibitem{Gould1987}  M.D. Gould, J. Austral. Math. Soc. Ser. B {\bf 28}  (1987) 310.
%  
% \bibitem{GT1950}
% I.M. Gelfand and M.L. Tsetlin, Dokl. Akad. Nauk., SSSR {\bf 71} (1950) 825 (Russian).
% English translation in: I.M. Gelfand, ``Collected Papers'', Vol II, Berlin:
% Springer-Verlag (1988) 653.
%  
% \bibitem{GT1950b}
% I.M. Gelfand and M.L. Tsetlin, Dokl. Akad. Nauk., SSSR {\bf 71} (1950) 1017 (Russian).
% English translation in: I.M. Gelfand, ``Collected Papers'', Vol II, Berlin:
% Springer-Verlag (1988) 657.
%  
% \bibitem{BB1963} G.E. Baird and L.C. Biedenharn, J. Math. Phys. {\bf 4} (1963) 1449.
%  
% \bibitem{Palev1987} T. D. Palev, Funct. Anal. Appl. {\bf 21} (1987) 245.
%  
% \bibitem{Palev1989} T. D. Palev, Funct. Anal. Appl. {\bf 23} (1989) 141.
% 
% \bibitem{StoiVan2010} N.I. Stoilova and J. Van der Jeugt, J. Math. Phys. {\bf 51} (2010)
%  093523.
% 
% \bibitem{Molev2011} A.I. Molev, Bull. Inst. Math. Acad. Sinica {\bf 6} (2011) 415.
% 
% \bibitem{TolIstSmi1986} V.N. Tolstoy, I.F. Istomina and Yu.F. Smirnov, in ``Group
% Theoretical Methods in Physics: Proceedings of the Third Yurmala Seminar'', Yurmala, USSR,
% 1985, Ed. M.A. Markov, V.I. Man'ko and V.V. Dodonov, VNU Science Press, Utrecht (1986)
% 337.
%  
%\bibitem{Kac1977} V.G. Kac, Adv. in Math. {\bf 26} (1977) 8.
%% 
%\bibitem{Kac1978} V.G. Kac, Lecture Notes in Math. {\bf 676}, Springer, Berlin (1978) 597.
%% 
%\bibitem{NahmSch1976} W. Nahm and M. Scheunert, J. Math. Phys. {\bf 17} (1976) 868.
%% 
%\bibitem{GouZha19902} M.D. Gould and R.B. Zhang, J. Math. Phys. {\bf 31} (1990) 1524.
%% 
%\bibitem{GouBraHug1989} M.D. Gould, A.J. Bracken and J.W.B. Hughes, J. Phys. A: Math. Gen.
%{\bf 22} (1989) 2879.
% 
%\bibitem{GouJarBra1990} M.D. Gould, P.D. Jarvis and A.J. Bracken, J. Math. Phys. {\bf 31}
%(1990) 2803. 
%% 
% \bibitem{PaStVa1994} T.D. Palev, N.I. Stoilova and J. Van der Jeugt, Comm. Math. Phys.
% {\bf 166} (1994) 367.
% 
%\bibitem{Gould1992}
%M.D. Gould, J. Math. Phys. {\bf 33} (1992) 1023.
%% 
%\bibitem{Gould1981}
%M.D. Gould, J. Math. Phys. {\bf 22} (1981) 15.

% \bibitem{SchNaRit1976} M. Scheunert, W. Nahm and V. Rittenberg, J. Math. Phys. {\bf 17}
% (1976) 1626.
% 
% \bibitem{Sch1979} M. Scheunert, Lecture Notes in Math. {\bf 716}, Springer, Berlin (1979).
% 
% \bibitem{Ram1971} P. Ramond, Phys. Rev. D {\bf 3} (1971) 2415.
% 
% \bibitem{NevSchw1971} A. Neveu and J.H. Schwarz, Nucl. Phys. B {\bf 31} (1971) 86.
% 
% \bibitem{VolAk1973} D.V. Volkov and V.P. Akulov, Phys. Lett. B {\bf 46} (1973) 109.
% 
% \bibitem{WessZum1974} J. Wess and B. Zumino, Nucl. Phys. B {\bf 70} (1974) 39.
% 
% \bibitem{SalStrath1974} A. Salam and J. Strathdee, Nucl. Phys. B {\bf 76} (1974) 477.
% 
% \bibitem{Scherk1975} J. Scherk, Rev. Mod. Phys. {\bf 47} (1975) 123.
% 
% \bibitem{FayFer1977} P. Fayat and S. Ferrara, Phys. Rep. {\bf 32} (1977) 249.
% 
% \bibitem{CorNeSt1975} L. Corwin, Y. Ne'eman and S. Sternberg, Rev. Mod. Phys. {\bf 47}
% (1975) 573.
% 
% \bibitem{Musson2012} I.M. Musson, Graduate Studies in Math. {\bf 131}, AMS (2012).
% 
% \bibitem{BeiStau2003} N. Beisert and M. Staudacher, Nucl. Phys. B {\bf 670} (2003) 439.
% 
% \bibitem{Mina2012} J.A. Minahan, Lett. Math. Phys. {\bf 99} (2012) 33.
% 
% \bibitem{GalMar2004} W. Galleas and M.J. Martins, Nucl. Phys. B {\bf 699} (2004) 455.
% 
% \bibitem{EssFraSal2005} F.H.L. Essler, H. Frahm and H. Saleur, Nucl. Phys. B {\bf 712}
% (2005) 513.
% 
% \bibitem{ZhYaZh2006} S.Y. Zhao, W.L. Yang and Y.Z. Zhang, Commun. Math. Phys. {\bf 268}
% (2006) 505.
% 
% \bibitem{RagSat2007} E. Ragoucy and G. Satta, JHEP09 (2007) 001.
% 
% \bibitem{FraMar2011} H. Frahm and M.J. Martins, Nucl. Phys. B {\bf 847} [FS] (2011) 220.
% 
% \bibitem{SchomSal2006} V. Schomerus and H. Saleur, Nucl. Phys. B {\bf 734} [FS] (2006) 221.
% 
% \bibitem{Ridout2009} D. Ridout, Nucl. Phys. B {\bf 810} [FS] (2009) 503.
% 
% \bibitem{StoiVan2008} N.I. Stoilova and J. Van der Jeugt, J. Phys. A: Math. Theor. {\bf
% 41} (2008) 075202.
% 
% \bibitem{LievStoiVan2008} S. Lievens, N.I. Stoilova and J. Van der Jeugt, Commun. Math.
% Phys. {\bf 281} (2008) 805.
% 
% 
% \bibitem{Molev2006} A.I. Molev, Handbook of Algebra {\bf 4} (2006) 109.
 
% 
% \bibitem{KamKyPal1989} A.H. Kamupingene, N.A. Ky, T.D. Palev, J. Math. Phys. {\bf 30}
% (1989) 553.
% 
% 
% \bibitem{PalStoi1990} T.D. Palev, N.I. Stoilova, J. Math. Phys. {\bf 31} (1990) 953.
% 
% 
% \bibitem{Dirac1936} P.A.M. Dirac, Proc. R. Soc. Lond. A {\bf 155} (1936) 447.
% 
% \bibitem{BB1964b} G.E. Baird and L.C. Biedenharn, J. Math. Phys. {\bf 5} (1964) 1723.
% 
\bibitem{Gould1978} M.D. Gould, J. Austral. Math. Soc. Ser. B {\bf 20}  (1978) 401.
% 
\bibitem{Gould1980} M.D. Gould, J. Math. Phys. {\bf 21} (1980) 444.
% 
% 
% \bibitem{Gould1981b}
% M.D. Gould, J. Math. Phys. {\bf 22} (1981) 2376.
% 
% \bibitem{Gould1984} M.D. Gould, J. Phys. A: Math. Gen. {\bf 17} (1984) 1.
% 
% \bibitem{Kostant1975} B. Kostant, J. Func. Anal {\bf 20} (1975) 257.
% 
% \bibitem{Hannabuss1997} K. Hannabuss, ``An Introduction to Quantum Theory'', Oxford
% University Press, Oxford (1997).
% 
% \bibitem{BB1964} G.E. Baird and L.C. Biedenharn, J. Math. Phys. {\bf 5} (1964) 1730.
% 
% \bibitem{LouBei1970} J.D. Louck and L.C. Biedenharn, J. Math. Phys. {\bf 11} (1970) 2368.
% 
% \bibitem{ZhGoBr1991} R.B. Zhang, M.D. Gould and A.J. Bracken, Nucl. Phys. B {\bf 354}
% (1991) 625.
% 
% \bibitem{RitSch1992} V. Rittenberg and M. Scheunert, J. Math. Phys. {\bf 33} (1992) 436.
% 
% \bibitem{Mozr2005} M. Mozrzymas, Int. J. Geom. Meth. Mod. Phys. {\bf 2} (2003) 393.
% 
% \bibitem{PaisRitt1975} A. Pais and V. Rittenberg, J. Math. Phys. {\bf 16} (1975) 2062.
% 
% \bibitem{Mezin1977} L. Mezincescu, J. Math. Phys. {\bf 18} (1977) 453.
% 
% \bibitem{Mozr2004} M. Mozrzymas, J. Phys. A: Math. Gen. {\bf 37} (2004) 9515.
% 
% \bibitem{JarMur1983} P.D. Jarvis and M.K. Murray, J. Math. Phys. {\bf 24} (1983) 1705.
% 
% \bibitem{JaRuYa2011}
% P.D. Jarvis, G. Rudolph and L.A. Yates, J. Phys. A: Math. Theor. {\bf 44} (2011) 235205.

%\bibitem{ClarkPeng2015}
%S. Clark, Y.N. Peng and S.K. Thamrongpairoj, Linear Multilinear A. {\bf 63} (2015) 274.
% 
% 


\end{thebibliography}


\end{document}
